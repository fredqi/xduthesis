%% ----------------------------------------------------------------------
%% START OF FILE
%% ----------------------------------------------------------------------
%% 
%% Filename: biography.tex
%% Author: Fred Qi
%% Created: 2012-12-14 17:52:29(+0800)
%% 
%% ----------------------------------------------------------------------
%%% CHANGE LOG
%% ----------------------------------------------------------------------
%% Last-Updated: 2016-02-09 15:16:51(-0700) [by Fred Qi]
%%     Update #: 22
%% ----------------------------------------------------------------------

\begin{cauthorbio}{\includegraphics[height=1.25in]{photo}}
  某某某,某省某市人。XXXX年毕业于某某某大学、获学士学位。简要介绍攻读硕士、博士学位
  经历。主要研究方向:略。指导老师:某某某~教授。

  代表性成果及经历:(获奖、专利、专著、论文等信息、参与或完成实际工程、访学经历)。

  2012年版研究生学位论文中要求提供学位申请人的中英文科研简历,并提供照片。
  在2014年修改后无此要求。排版学术简历的环境在模板中保留了下来,供必要的时候使
  用。

  用于排版中英文简历的环境分别为
  \begin{itemize}
  \item \texttt{cauthorbio}
  \item \texttt{eauthorbio}.
  \end{itemize}
  需要注意的是中文简历需要提供作者脸部照片。具体使用方法
  见\texttt{examples/biography.tex}。
  % \vspace{1cm}
\end{cauthorbio}

\begin{eauthorbio}  
  \hspace{0.6cm} Author biography in English.

  There is a requirement to provide the author's biography with a portrait photo
  in the thesis writing guidelines published in 2012. But this requirement was
  removed in the guidelines updated in 2014. However, the environments for
  typesetting the author's biography was kept in the template for later use.

  The environments designed for biography are
  \begin{itemize}
  \item \texttt{cauthorbio}
  \item \texttt{eauthorbio}.
  \end{itemize}
  Please note that the two environment are designed differently since the
  Chinese version are with the author's portrait photo. An example is provided
  in the file \texttt{examples/biography.tex}.
\end{eauthorbio}

%% ----------------------------------------------------------------------
%%% END OF FILE 
%% ----------------------------------------------------------------------