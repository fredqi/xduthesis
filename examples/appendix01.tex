%% ----------------------------------------------------------------------
%% START OF FILE
%% ----------------------------------------------------------------------
%% 
%% Filename: appendix01.tex
%% Author: Fred Qi
%% Created: 2013-01-18 20:20:46(+0800)
%% 
%% ----------------------------------------------------------------------
%%% CHANGE LOG
%% ----------------------------------------------------------------------
%% Last-Updated: 2014-11-27 22:01:02(+0300) [by Fred Qi]
%%     Update #: 31
%% ----------------------------------------------------------------------


\chapter[2014年版研究生学位论文撰写要求]{西安电子科技大学研究生学位论文撰写要
  求\\(2014年修订)}

学位论文是体现作者从事科学研究取得的创造性成果和创新见解,并以此为内容撰写的、作
为提出申请授予相应的学位评审用的学术论文。为统一规范我校研究生学位论文撰写和编辑
格式,按照国家标准《学位论文编写规则》 (GB/T 7713.1--2006)规定,现对我校研究生
学位论文撰写要求如下:

\section{研究生学位论文撰写的总体要求}

研究生学位论文必须是学位申请者本人在导师的指导下独立完成的学术成果,要求立论正确,
推理严谨,数据可靠,层次分明,文字通畅,不得抄袭和剽窃他人成果。研究生学位论文的
撰写语言为中文和英文两种,硕士学位论文篇幅一般不低于3万字,博士学位论文篇幅一般
不低于5万字。

研究生学位论文中使用的术语、符号、代号必须全文统一并符合规范化要求,计量单位一律
采用国务院1984年2月27日发布的《中华人民共和国法定计量单位》标准。

\section{研究生学位论文撰写的内容要求}

我校研究生学位论文包括以下几个部分:

\subsection{封面}

\begin{enumerate}
\item 题目:题目是以最恰当、最简明的词语反映论文中最重要的特定内容的逻辑组合,力
  求简短切题。中文题目(包括副标题和标点符号)一般不超过20个字,英文题目一般不超
  过10个实词。
\item 责任者姓名:包括论文作者姓名、指导教师姓名及职称(博士学位论文、学术型和同
  等学力硕士学位论文)以及学校、企业导师姓名及职称(专业学位硕士学位论文)。
\item 一级学科和二级学科(博士学位论文、学术型和同等学力硕士学位论文):学科名称
  的填写必须严格按照国务院学位委员会1997年颁布的《授予博士、硕士学位和培养研究生
  的学科、专业目录》的规定填写。
\item 领域(专业学位硕士学位论文):按照专业学位领域填写,如电子与通信工程、计算
  机技术、机械工程等。
\item 申请学位类别:按照学科门类和学位层次填写,如工学博士、工学硕士、工程硕士、
  工商管理硕士等。
\item 提交学位论文日期:填写学位论文送审的日期。
\end{enumerate}

学位论文题目居于封面空白处中间位置,字体为宋体,字号为二号加粗。作者姓名、导师姓
名职称、一级学科、二级学科、领域、申请学位类别、提交学位论文日期的标题字体为黑体,
字号为四号加粗,具体内容的字体为宋体,字号为四号加粗。博士学位论文封面中缝字体为
黑体,字号为小四号加粗。

\subsection{题名页}

包括中文题名页和英文题名页,主要由学校代码、分类号、学号、密级、“论文题目”、“作
者姓名”、“一级学科”、“二级学科”(博士学位论文、学术型和同等学力硕士学位论文)、
“领域”(专业学位硕士学位论文)、“申请学位类别”、“指导教师姓名、职称”(博士学位论
文、学术型和同等学力硕士学位论文)、“学校、企业导师姓名、职称”(专业学位硕士学位
论文)以及“论文提交日期”等部分组成。撰写方式见《西安电子科技大学博士/硕士/专业学
位硕士学位论文模板》范例。 

\begin{enumerate}
\item 学校代码:指本单位编号,我校代码是“10701”。
\item 分类号:指在《中国图书资料分类法》中的分类号(填写前四位即可),可在校图书
  馆网站上查询。
\item 学号:按照入学时研究生院编制的统一编号填写。
\item 密级:密级由导师确定,分为公开和秘密两种。
\end{enumerate}

中文题名页中的学校代码、分类号、学号和密级的字体为宋体,字号为五号加粗;学位论文
题目字体为宋体,字号为二号加粗;作者姓名、导师姓名职称、一级学科、二级学科、领域、
申请学位类别、提交学位论文日期的标题和具体内容的字体为宋体,标题字号为四号加粗,
具体内容的字号为四号不加粗。

英文题名页中的学科填写一级学科(专业学位填写领域),学位论文题目字体为Times New
Roman,字号二号加粗,其他内容的字体为Times New Roman,字号三号。

\subsection{声明}

是对学位论文创新性和使用授权的声明和说明,论文提交图书馆和存档时作者本人和指导教
师必须签名确认。

\subsection{摘要}

摘要是学位论文的内容不加注释和评论的简短陈述,简明扼要陈述学位论文的研究目的、内
容、方法、成果和结论,重点突出学位论文的创造性成果和观点。摘要包括中文摘要和英文
摘要,中文摘要力求精炼准确,硕士学位论文中文摘要字数一般为1000字左右,博士学位论
文中文摘要字数一般为1500字左右。英文摘要内容与中文摘要内容保持一致,翻译力求简明
精准。摘要页的下方注明论文的关键词和论文类型,关键词一般为3~5个,关键词和关键词
之间用逗号并空一格;论文类型一般分为2种:应用基础研究类和基础研究类。全日制工程
硕士学位论文类型分为8种:工程(规划)设计、调研报告、应用基础技术、实用新型技术、
应用软件技术、技术报告、工程(项目)管理和案例分析以及技术论文,具体要求见《西安
电子科技大学全日制工程硕士学位论文写作要求及范文》。

中文摘要标题字体为宋体,字号为三号加粗,上下各空一行;正文字体为宋体,字号为小四
号;关键词和正文之间空一行,关键词和论文类型字体为宋体,字号为五号,标题加粗。英
文摘要标题字体为Times New Roman,字号为三号加粗,上下各空一行;正文的每一段落首
行不空格,段落与段落之间空一行;正文字体为Times New Roman,字号为小四号;关键词
和论文类型字体为Times New Roman,字号为五号,标题加粗。

\subsection{插图索引}

学位论文中插图的目录索引。插图索引标题字体为宋体,字号为三号加粗,上下各空一行;
正文内容字体为宋体,字号为小四号。

\subsection{表格索引}

学位论文中表格的目录索引。表格索引标题字体为宋体,字号为三号加粗,上下各空一行;
正文内容字体为宋体,字号为小四号。

\subsection{符号对照表}

学位论文中符号代表的意义及单位(或量纲)的说明。符号对照表标题字体为宋体,字号为
三号加粗,上下各空一行;正文内容字体为宋体,字号为小四号。

\subsection{缩略语对照表}

学位论文中缩略语代表意义的说明。缩略语对照表标题字体为宋体,字号为三号加粗,上下
各空一行;正文内容中文字体为宋体,字号为小四号,英文字体为Times New Roman,字号
为小四号。

\subsection{目录}

目录是学位论文的提纲,是论文各组成部分的小标题,应分别依次列出并注明页码。各级标
题分别以第一章、1.1、1.1.1等数字依次标出,目录中最多列出三级标题,正文中如果确需
四级标题,用1)、2)形式标出。学位论文的前置部分(摘要、插图索引、表格索引、符号
对照表、缩略语对照表)和学位论文的主体部分(正文、参考文献、致谢、作者简介)都要
在目录中列出。

目录标题字体为宋体,字号为三号加粗,上下各空一行;目录内容字体为宋体,字号为小四
号。

\subsection{正文}

正文是学位论文的主体和核心部分。正文的一级标题居中排列,字体为宋体,字号为三号加
粗,上下各空一行;二级标题段前不空格,字体为宋体,字号为小三号,段前设置为“自动”;
三级标题段前空2字符,字体为宋体,字号为四号;正文内容字体为宋体,字号为小四号。
正文一般包括以下几个方面:

\paragraph{1.绪论}

绪论是学位论文主体部分的开端,切忌与摘要雷同或成为摘要的注解。绪论除了要说明论文
的研究目的、研究方法和研究结果外,还应评述与论文研究内容相关的国内外研究现状和相
关领域中已有的研究成果;其次还要介绍本项研究工作的前提和任务、理论依据、实验基础、
涉及范围、预期结果以及该论文在已有基础上所要解决的问题。

\paragraph{2.各章节}

各章节一般由标题、文字叙述、图、表、公式等构成,章节内容总体要求立论正确,逻辑清
晰,数据可靠,层次分明,文字通畅,编排规范。论文中若有与指导教师或他人共同研究的
成果,必须明确标注;如果引用他人的结论,必须明确注明出处,并与参考文献保持一致。

(1)图:包括曲线图、示意图、流程图、框图等。图序号一律用阿拉伯数字分章依序编码,
如:图1.3、图2.11。每一个图应有简短确切的图名,连同图序号置于图的正下方。图名称
字体为宋体,字号为五号。图中坐标上标注的符号和缩略词必须与正文保持一致。引用图应
在图题右上角标出文献来源;曲线图的纵横坐标必须标注“量、标准规定符号、单位”,这三
者只有在不必要标明(如无量纲等)的情况下方可省略。

(2)表:包括分类项目和数据,一般要求分类项目由左至右横排,数据从上到下竖列。分
类项目横排中必须标明符号或单位,竖列的数据栏中不要出现“同上”、“同左”等词语,一律
要填写具体的数字或文字。表序号一律用阿拉伯数字分章依序编码,如:表2.5、表10.3。
每一表应有简短确切的题名,连同表序号置于表的正上方。表名称字体为宋体,字号为五号。

(3)公式:正文中的公式、算式、方程式等必须编排序号,序号一律用阿拉伯数字分章依序
编码,如:(3-32)、(6-21)。对于较长的公式,另起行居中横排,只可在符号处(
如:+、-、*、/、$<$ $>$等)转行。公式序号标注于该式所在行(当有续行时,应标注于最后一
行)的最右边。连续性的公式在“=”处排列整齐。大于999的整数或多于三位的小数,一律用
半个阿拉伯数字符的小间隔分开;小于1的数应将0置于小数点之前。

(4)计量单位:学位论文中出现的计量单位一律采用国务院1984年2月27日发布的《中华人
民共和国法定计量单位》标准。

\paragraph{3. 结论}
是学位论文最终和总体的结论,不是正文中各段的小结的简单重复,应准确、精炼、完整,
其中要着重阐述作者研究的创造性成果以及在本研究领域中的重大意义,还可提出有待进一
步研究和探讨的问题。

\subsection{参考文献}

参考文献是文中引用的有具体文字来源的文献集合,博士学位论文参考文献一般不少于80篇,
硕士学位论文参考文献一般不少于30篇。参考文献标题字体为宋体,字号为三号加粗,上下
各空一行;参考文献若是中文文献,字体为宋体,字号为五号,若是英文文献,字体为
Times New Roman,字号为五号。学位论文的撰写要本着严谨求实的科学态度,凡有引用他
人成果之处,引用处右上角用方括号标注阿拉伯数字编排的序号(必须与参考文献一致),
同时所有引用的文献必须用全称,不能缩写,并按论文中所引用的顺序列于文末。引用文献
的作者不超过3位时全部列出,超过时列前3位,后加“等”字或“et al.”。参考文献的著录要
符合《文后参考文献著录规则》(GB/T7714-2005)要求:

\begin{enumerate}
\item 期刊(报纸)参考文献:[序号] 主要责任者.文献名称[文献类型标
  志].期刊(报纸)名,年份, 卷(期):引文页码.
\item 专著参考文献:[序号] 主要责任者.专著名称[文献类型标志].其他责任者.出版地:出
  版单位,出版年份:引文页码.
\item 专利参考文献:[序号] 主要责任者.专利名称[文献类型标志].国别,专利种类,专利
  号,出版日期.
\item 技术标准参考文献:[序号] 起草责任者.标准代号-标准顺序号-发布年.标准名称[文
  献类型标志].出版地:出版单位,出版年份.
\item 电子参考文献:[序号] 主要责任者.题名[文献类型标志].[引用日期].获取和访问路
  径.
\item 会议论文集参考文献:[序号] 析出责任者.析出题名[文献类型标志]//编者.论文
  集名.(供选择项:会议名,会址,开会年)出版地:出版者,出版年份:引文页码.
\item 学位论文参考文献:[序号] 主要责任者.文献题名[文献类型标志].保存地:
  保存单位,年份.
\item 国际、国家标准参考文献:[序号] 标准代号.标准名称[文献类型标志].出版
  地:出版者,出版年.
\item 报告类参考文献:[序号] 主要责任者.文献题名[文献类型标志].报告地:报告
  会主办单位,年份.
\end{enumerate}

参考文献著录中的文献类别代码:
\begin{enumerate}
\item 普通图书:M
\item 会议录:C
\item 汇编:G
\item 报纸:N
\item 期刊:J
\item 学位论文:D
\item 报告:R
\item 标准:S
\item 专利:P
\item 数据库:DB
\item 计算机程序:CP
\item 电子公告:EB
\end{enumerate}

参考文献的具体著录方式参见《西安电子科技大学博士/硕士/专业学位硕士学位论文模板》
范例。

\subsection{致谢}

作者对完成论文提供帮助和支持的组织和个人表示感谢的文字记载。

\subsection{作者简介}

对作者的简要介绍,主要包括个人基本情况、教育背景、攻读博士/硕士学位期间的研究成
果等三个部分内容。攻读博士/硕士学位期间的研究成果是指本人攻读博士/硕士学位期间发
表(或录用)的学术论文,获得的科研成果、专利以及参与的科研项目等,分别按时间顺序
列出。其中,发表论文、发明专利、科研获奖只列出作者排名前3名的,参与的科研项目按
重要程度最多列出5项。此部分内容严格按照《西安电子科技大学博士/硕士/专业学位硕士
学位论文模板》范例书写。

\subsection{其他}

学位论文中如果需要注释,可作为脚注在页下分别著录,切忌在文中注释;如果有附录部分,
可编写在正文之后,与正文连续编页码,每一附录均另页起,附录依次用大写英文字母A、B、
C……编序号,如:附录A、附录B等。

\section{ 研究生学位论文的编辑、打印、装订要求}

\subsection{学位论文封面的编辑和打印要求}

学位论文的封面由研究生院按国家规定统一制定印刷,封面内容必须打印,不得手写。

\subsection{学位论文的版面设置要求}

\begin{enumerate}
\item 行间距:固定值20磅(题名页除外);
\item 字符间距:标准;
\item 页眉设置:单面页码页眉标题为章节题目,每一章节的起始页必须在单面页码,双面
  页码页眉标题统一为“西安电子科技大学博/硕士学位论文”,页眉标题居中排列,字体为
  宋体,字号为五号。
\item 页码设置:学位论文的前置部分和主体部分分开设置页码,前置部分的页码用罗马数
  字标识,字体为Times New Roman,字号为小五号;主体部分的页码用阿拉伯数字标识,
  字体为宋体,字号为小五号。页码统一居于页面底端中部,不加任何修饰;
\item 页面设置:为了便于装订,要求每页纸的四周留有足够的空白边缘,其中页边距为
  上3厘米、下2厘米;内侧3厘米、外侧2厘米;装订线为1厘米;页眉2厘米,页脚1.75厘
  米。
\end{enumerate}

\section{学位论文的打印、装订要求}

\begin{enumerate}
\item 打印:学位论文必须用A4纸页面排版,双面打印;
\item 装订:依次按照中文题名页、英文
  题名页、声明、摘要、插图索引、表格索引、符号对照表、缩略语对照表、目录、正文、
  附录(可选)、参考文献、致谢、作者简介的顺序,用学校统一印制的学位论文封面装订
  成册。盲审论文必须删除致谢部分以及封面和研究成果中的作者和指导教师姓名。
\end{enumerate}
\section{其他说明}

\begin{enumerate}
\item 学位论文模板为简单范例,仅供参考,论文的编辑、排版、打印等以本撰写要求为
  准;
\item 本规定由研究生院负责解释,从申请2014年12月毕业和授位的研究生开始执行,其它
  有关规定同时废止。学位申请人员必须严格按照本规定撰写学位论文,不遵照本规定
  者, 一律不予送审学位论文。研究生毕业论文撰写要求参照学位论文撰写要求执行。
\end{enumerate}

\begin{flushright}
  研究生院\\ 2014年11月
\end{flushright}

\chapter{公式推导}

\section{一些常用公式}
\label{sec:someeq}

$$ c^{2} = a^{2} + b^{2} $$

\section{一些重要公式}
\label{sec:importanteq}

$$ E=mc^{2}$$

$$ e^{i\theta} = \frac{cos\theta + i \sin\theta}{2} $$


%% ----------------------------------------------------------------------
%%% END OF FILE 
%% ----------------------------------------------------------------------